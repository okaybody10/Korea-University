\documentclass[12pt]{article}

\title{Probability Theory}
\author{Sung Jae Hyuk\\Majoring in Computer Science $\&$ Mathematics\\Korea University}
\date{}

\usepackage{indentfirst}
\usepackage[shortlabels]{enumitem}
\usepackage{kotex}
\usepackage{amsmath, amssymb, amsthm, amsfonts, graphics, epsfig, fancyhdr}
\setlength{\headheight}{28pt}
\pagestyle{fancy}
\fancyhf{}

\fancyhead[R]{Sung Jae Hyuk\\ Probability theory, HW1}
\fancyfoot[C]{\thepage}

\setlength\parindent{12pt}
\theoremstyle{plain}
\newtheorem{theorem}{Theorem}
\newtheorem{lemma}[theorem]{Lemma}
\newtheorem{corollary}[theorem]{Corollary}
\newtheorem{proposition}[theorem]{Proposition}
\newtheorem*{definition}{Definition}

\renewcommand{\qedsymbol}{\ensuremath{\blacksquare}}
\newcommand{\N}{\mathbb{N}}
\newcommand{\Z}{\mathbb{Z}}
\newcommand{\Q}{\mathbb{Q}}
\newcommand{\R}{\mathbb{R}}
\newcommand{\C}{\mathbb{C}}

\begin{document}
	\maketitle
	\newpage
\section*{Problem 1}
	\subsection*{Statement}
	Suppose $8$ identical blackboards are to be divided among $4$ schools.
	\begin{enumerate}[(a)]
		\item How many divisions are possible?
		\item How many if each school must receive at least $1$ blackboard?
	\end{enumerate}
	\subsection*{Solution}
	\begin{enumerate}[(a)]
		\item  Every blackboards can be divided any school.
		\\Hence, the total number of divisions is $\underbrace{4\times4\times\cdots\times4}_{8\ \text{blackboards}}=4^8$
		\item  Let us use inclusion-exclusion principle.
		\\ First, There are $4^8$ ways on which all schools receive blackboards.
		\\ Second, There are $3^8\times\binom{4}{1}$ ways on which only one school doesn't receive blackboard.
		\\ Third, There are $2^8\times\binom{4}{2}$ ways on which two schools don't recevie blackboard.
		\\ Lastly, There are $\binom{4}{3}$ ways on which only one school receives blackboard.
		\\ Hence, the total number of cases is
		$$4^8-4\times3^8+6\times2^8-4\times1=40824$$
	\end{enumerate}
\newpage
\section*{Problem 2}
\subsection*{Statement}
A $5$-card hand is dealt from a well-shuffled deck of $52$ playing cards. What is the probability that the hand contains at least one card from each of the four suits?
\subsection*{Solution}
First we know that there are $13$ cards for each suits.
\par Thus, we can evaluate the probability by choosing a suit to be contained twice, and choosing a card for each suit.
\par The denominator is $\binom{52}{5}$, and the numerator is product of the number of ways of choosing one among four suits and choosing a card from each suits.
\par Because first is $4$ and second is $13^3\times \binom{13}{2}$, probability is
$$\begin{aligned}
	\dfrac{4\times 13^3\times \binom{13}{2}}{\binom{52}{5}}&=\dfrac{4\times13^3\times13\times12\times5!}{52\times51\times50\times49\times48\times2!}
	\\&=\dfrac{2197}{8330}\approx0.26
\end{aligned}$$
\newpage
\section*{Problem 3}
\subsection*{Statement}
From a group of $3$ first-year students, $4$ sophomores, $4$ juniors, and $3$ seniors, a committee of size $4$ is randomly selected. Find the probability that the committee will consist of
\begin{enumerate}[(a)]
	\item $1$ from each class
	\item $2$ sophomores and $2$ juniors
	\item Only sophomores or juniors
\end{enumerate}
\subsection*{Solution}
First, there are $\binom{14}{4}$ ways of selecting student randomly.
\begin{enumerate}[(a)]
	\item Let $n_1,n_2,\cdots,n_k$ be the group sizes.
	\\ If we select only one student each group, then there are $n_1\times n_2\times \cdots n_k$ ways by product rule.
	\\ Hence there are $3\times 4\times 4\times 3=144$ ways of selecting member, the probability is $\dfrac{144}{1001}\approx0.14$
	\item There are $\binom{4}{2}$ ways of selecting $2$ sophomores, and $\binom{4}{2}$ ways of selecting $2$ juniors.
	\\ As $\binom{4}{2}=6$, there are $6^2=36$ ways on which committee will consist of $2$ sophomores and $2$ juniors, the probability is $\dfrac{36}{1001}\approx0.04$.
	\item There are $8$ students in sophomores and juniors, hence The number of choosing committee member is $\binom{8}{4}$
	\\ Thus, the probability is $\dfrac{70}{1001}=\dfrac{10}{143}\approx0.07$
\end{enumerate}
\newpage
\section*{Problem 4}
\subsection*{Statement}
A hospital administrator codes incoming patients suffering gunshot wounds according to whether they have insurance (coding $1$ if they do and $0$ if they do not) and according to their condition, which is rated as good (g), fair (f), or serious (s). Consider an experiment that consists of the coding of such a patient.
\begin{enumerate}[(a)]
	\item Give the sample space of this experiment
	\item Let $A$ be the event that the patient is in serious condition. Specify the outcomes in $A$
	\item Let $B$ be the event that the patient is uninsured. Specify the outcomes in $B$
	\item Give all the outcomes in the event $B^c\cup A$
\end{enumerate}
\subsection*{Solution}
	Let $(a,~b)$ be a pair where $a$ is whether they have insuarnce, $b$ is condition which is rated as g, f, s.
\begin{enumerate}[(a)]
		\item $\Omega=\left\{(1,~g),~(1,~f),~(1,~s)~(0,~g),~(0,~f),~(0,~s)\right\}$
		\item As patient is in serious condition, second part of pair is s
		\\ Thus $(\text{outcomes in A})=\left\{(1,~s),~(0,~s)\right\}$
		\item As patient is uninsured, first part of pair is $0$
		\\ Thus $(\text{outcomes in B})=\left\{(0,~f),~(0,~g),~(0,~s)\right\}$
		\item Since $B=\left\{(0,~f),~(0,~g),~(0,~s)\right\}$, $B^c=\left\{(1,~f),~(1,~g),~(1,~s)\right\}$
		\\ Thus $B^c\cup A$ is
		$$\left\{(1,~f),~(1,~g),~(1,~s),~(0,~s)\right\}$$
\end{enumerate}

\newpage
\section*{Problem 5}
\subsection*{Statement}
Suppose that Seoul-metro consists of $n$ cars, and $k\ (k\geq n)$ passengers get on it by selecting one of their cars randomly. Find the probability that there will be at least one passenger in each car.
\subsection*{Solution}
For convenience, suppose that the cars have an index $i=1,2,...,n$
\\Let $A$ be the event that there will be at least one passenger in each car. Then, $A=\cap_{i=1}^n A_i$ where $A_i$ denotes the event at least one passenger in $i$-th car.
\\ Suppose $B=A^c$, then $B=A^c=\cup_{i=1}^n A_i^c=\cup_{i=1}^n B_i$ where $B_i$ denotes the event that there will be no passenger in car $i$
\\ We want to evaluate $P\left(A\right)$, but it is difficult to get value directly. Thus we will evaluate $P(B)$ instead of $P(A)$, and will use the formula $P(A)+P(B)=1$
\\ To determine $P(B)$, we will use include-exclusion principle. Note that each outcome is equally likely, so the probability of each set is $1/{n^k}$ times the number of outcomes in the set.
\\ Each $B_i$ has those outcomes in which there will be no passenger in $i$-th car. Because there are $(n-1)$ ways for each passenger, the nubmer of cases is $(n-1)^k$, so $P(B_i)=(n-1)^k/n^k $ for each i. 
\\ Each $B_i\cap B_j$ has those outcomes in which there will be no passenger in not only $i$-th car but $j$-th car. The number of ways is $(n-2)^k$, hence $P(B_i\cap B_j)=(n-2)^k/n^k$
\\ In same way, we see that $P(B_1\cap B_2\cap \cdots\cap B_\alpha)=(n-\alpha)^k/n^k$, so we can compute $P(\cup_{i=1}^n B_i)$ using inclusion-exclusion principle, result is
$$\begin{aligned}
	P(\cup_{i=1}^n B_i)&=\displaystyle{\sum_{i=1}^{n}P(B_i)}-\displaystyle{\sum_{i<j}P(B_i\cap B_j)+\cdots+(-1)^{n+1}P(\cap_{i=1}^nB_i)}
	\\&=\displaystyle{\sum_{i=1}^{n}(-1)^{n+1}\binom{n}{i}\left(1-\dfrac{i}{n}\right)^k}
\end{aligned}$$
\\Finally, $P(A)=1-P(B)=\displaystyle{\sum_{i=0}^n(-1)^n\binom{n}{i}\left(1-\dfrac{i}{n}\right)^k}\qquad\qquad\qquad\qedsymbol$

\newpage
\vspace*{\fill} 
\begin{quote} 
	\centering 
	This Page Intentionally Left Blank 
\end{quote}
\vspace*{\fill}
\end{document}